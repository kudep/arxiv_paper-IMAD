\documentclass{article}


\usepackage{PRIMEarxiv}

\usepackage[utf8]{inputenc} % allow utf-8 input
\usepackage[T1]{fontenc}    % use 8-bit T1 fonts
\usepackage{hyperref}       % hyperlinks
\usepackage{amsmath}
\usepackage{bbm}
\usepackage{url}            % simple URL typesetting
\usepackage{booktabs}       % professional-quality tables
\usepackage{amsfonts}       % blackboard math symbols
\usepackage{nicefrac}       % compact symbols for 1/2, etc.
\usepackage{microtype}      % microtypography
\usepackage{lipsum}
\usepackage{fancyhdr}       % header
\usepackage{graphicx}       % graphics
\graphicspath{{media/}}     % organize your images and other figures under 
\usepackage{import}




%Header
\pagestyle{fancy}
\thispagestyle{empty}
\rhead{ \textit{Moskvoretskii V. et al.}} 

% Update your Headers here
\fancyhead[LO]{IMAD}
%\fancyhead[RE]{Moskvoretskii V.} % Firstauthor et al. if more than 2 - must use \documentclass[twoside]{article}



  
%% Title
\title{IMAD: IMage-Augmented multi-modal Dialogue
%%%% Cite as
%%%% Update your official citation here when published 
%\thanks{\textit{\underline{Citation}}: 
%\textbf{Authors. Title. Pages.... DOI:000000/11111.}} 
}

\author{
  Moskvoretskii Viktor \\
  DeepPavlov.ai \\
  \texttt{vvmoskvoretskiy@yandex.ru} \\
  %% examples of more authors
   \And
  Frolov Anton \\
  DeepPavlov.ai \\
  \texttt{antonylovanto@gmail.com} \\
  \And
  Kuznetsov Denis \\
  DeepPavlov.ai \\
  \texttt{kuznetsov.den.p@gmail.com} \\

}


\begin{document}




\maketitle


\begin{abstract}
  Currently, dialogue systems have achieved high performance in processing text-based communication. However, they have not yet effectively incorporated visual information, which poses a significant challenge. Furthermore, existing models that incorporate images in dialogue generation focus on discussing the image itself. Our proposed approach presents a novel perspective on multi-modal dialogue systems, which interprets the image in the context of the dialogue. By doing so, we aim to expand the capabilities of current dialogue systems and transition them from single modality (text) to multi-modality. However, there is a lack of validated English datasets that contain both images and dialogue contexts for this task. Thus, we propose a two-stage approach to automatically construct a multi-modal dialogue dataset. In the first stage, we utilize text-to-image similarity and sentence similarity to identify which utterances could be replaced with an image. In the second stage, we replace those utterances by selecting a subset of relevant images and filtering them with a visual question answering model. We used this approach, along with additional labeling, to create the IMage Augmented multi-modal Dialogue dataset (IMAD), which can serve as a validated dataset for this task. Furthermore, we propose a baseline model trained on this dataset, which outperforms model trained on the same data without images and BlenderBot.
\end{abstract}


% keywords can be removed
\keywords{Natural Language Processing \and Deep Learning \and Machine Learning \and IMAD \and Dialogue Dataset \and Multi-modal Dataset \and Dialogue Systems \and Multi-modality.}


\section{Introduction}
\import{sections/}{intro.tex}

\section{Related Works}
\import{sections/}{RW.tex}

\section{Dialogue Datasets}
\import{sections/}{dialogue_datasets.tex}



\section{Text-Image Replacing}
\import{sections/}{replacing.tex}

\section{Multi-modal Dialogue Language Model}
\import{sections/}{Model.tex}


\section{Results and Discussion}
\import{sections/}{RD.tex}


\section{Conclusion}
\import{sections/}{Conclusion.tex}


% \section{Headings: first level}
% \label{sec:headings}

% \lipsum[4] See Section \ref{sec:headings}.

% \subsection{Headings: second level}
% \lipsum[5]
% \begin{equation}
% \xi _{ij}(t)=P(x_{t}=i,x_{t+1}=j|y,v,w;\theta)= {\frac {\alpha _{i}(t)a^{w_t}_{ij}\beta _{j}(t+1)b^{v_{t+1}}_{j}(y_{t+1})}{\sum _{i=1}^{N} \sum _{j=1}^{N} \alpha _{i}(t)a^{w_t}_{ij}\beta _{j}(t+1)b^{v_{t+1}}_{j}(y_{t+1})}}
% \end{equation}

% \subsubsection{Headings: third level}
% \lipsum[6]

% \paragraph{Paragraph}
% \lipsum[7]

% \section{Examples of citations, figures, tables, references}
% \label{sec:others}
% \lipsum[8] \cite{kour2014real,kour2014fast} and see \cite{hadash2018estimate}.

% The documentation for \verb+natbib+ may be found at
% \begin{center}
%   \url{http://mirrors.ctan.org/macros/latex/contrib/natbib/natnotes.pdf}
% \end{center}
% Of note is the command \verb+\citet+, which produces citations
% appropriate for use in inline text.  For example,
% \begin{verbatim}
%    \citet{hasselmo} investigated\dots
% \end{verbatim}
% produces
% \begin{quote}
%   Hasselmo, et al.\ (1995) investigated\dots
% \end{quote}

% \begin{center}
%   \url{https://www.ctan.org/pkg/booktabs}
% \end{center}


% \subsection{Figures}
% \lipsum[10] 
% See Figure \ref{fig:fig1}. Here is how you add footnotes. \footnote{Sample of the first footnote.}
% \lipsum[11] 

% \begin{figure}
%   \centering
%   \fbox{\rule[-.5cm]{4cm}{4cm} \rule[-.5cm]{4cm}{0cm}}
%   \caption{Sample figure caption.}
%   \label{fig:fig1}
% \end{figure}

% \subsection{Tables}
% \lipsum[12]
% See awesome Table~\ref{tab:table}.

% \begin{table}
%  \caption{Sample table title}
%   \centering
%   \begin{tabular}{lll}
%     \toprule
%     \multicolumn{2}{c}{Part}                   \\
%     \cmidrule(r){1-2}
%     Name     & Description     & Size ($\mu$m) \\
%     \midrule
%     Dendrite & Input terminal  & $\sim$100     \\
%     Axon     & Output terminal & $\sim$10      \\
%     Soma     & Cell body       & up to $10^6$  \\
%     \bottomrule
%   \end{tabular}
%   \label{tab:table}
% \end{table}

% \subsection{Lists}
% \begin{itemize}
% \item Lorem ipsum dolor sit amet
% \item consectetur adipiscing elit. 
% \item Aliquam dignissim blandit est, in dictum tortor gravida eget. In ac rutrum magna.
% \end{itemize}


% \section{Conclusion}
% Your conclusion here

% \section*{Acknowledgments}
% This was was supported in part by......

%Bibliography
\bibliographystyle{unsrt}  
\bibliography{references}  

\section*{Appendix A. Implementation details}
\import{sections/}{appendix_A.tex}

\section*{Appendix B. Dataset Reasoning}
\import{sections/}{appendix_B.tex}

%\section*{Appendix C. Finetuned models details}
%\import{sections/}{appendix_C.tex}


\section*{Appendix C. Training Details}
\import{sections/}{appendix_D.tex}

\section*{Appendix D. Labeling methodology}
\import{sections/}{appendix_E.tex}



\end{document}
